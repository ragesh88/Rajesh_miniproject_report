\documentclass[11pt]{report} % use larger type; default would be 10pt

\usepackage[utf8]{inputenc} % set input encoding (not needed with XeLaTeX)


%%% PAGE DIMENSIONS
\usepackage{geometry} % to change the page dimensions
\geometry{letterpaper} % or letterpaper (US) or a5paper or....
\geometry{margin=0.75in} % for example, change the margins to 2 inches all round
% \geometry{landscape} % set up the page for landscape
%   read geometry.pdf for detailed page layout information

%%% PACKAGES
\usepackage{datetime} % for date without day
\usepackage{graphicx} % support the \includegraphics command and options
\usepackage{epstopdf}
\usepackage{amsmath,amssymb,amsthm}
\usepackage{mathrsfs}                      % Gives us \mathscr for \setset
\usepackage{wrapfig}
\usepackage{array}
\usepackage[table]{xcolor}
\usepackage{booktabs}
\usepackage{titlesec}
\usepackage[normalem]{ulem}
\newcommand{\subfigureautorefname}{\figureautorefname}
\usepackage{chemarr}
\definecolor{Gray}{gray}{0.85}
\newdateformat{monthyeardate}{\monthname[\THEMONTH], \THEYEAR} % new date format
\newcommand{\ra}[1]{\renewcommand{\arraystretch}{#1}}
\newcolumntype{L}[1]{>{\raggedright\let\newline\\\arraybackslash\hspace{0pt}}m{#1}}
\newcolumntype{C}[1]{>{\centering\let\newline\\\arraybackslash\hspace{0pt}}m{#1}}
\newcolumntype{R}[1]{>{\raggedleft\let\newline\\\arraybackslash\hspace{0pt}}m{#1}}
\usepackage{indentfirst}
\usepackage[hidelinks]{hyperref}
\hypersetup{
	colorlinks,
	linkcolor={blue!50!black},
	citecolor={blue!50!black},
	urlcolor={blue!80!black}
}
% for writing algorithms
\usepackage{algpseudocode}
\usepackage{algorithm}

\makeatletter
\newcommand{\removelatexerror}{\let\@latex@error\@gobble}
\makeatother

\usepackage{booktabs} % for much better looking tables
\usepackage{array} % for better arrays (eg matrices) in maths
\usepackage{paralist} % very flexible & customisable lists (eg. enumerate/itemize, etc.)
\usepackage{verbatim} % adds environment for commenting out blocks of text & for better verbatim
\usepackage{subfig} % make it possible to include more than one captioned figure/table in a single float
% These packages are all incorporated in the memoir class to one degree or another...

%%% HEADERS & FOOTERS
\usepackage{fancyhdr} % This should be set AFTER setting up the page geometry
\pagestyle{fancy} % options: empty , plain , fancy
\renewcommand{\headrulewidth}{0pt} % customise the layout...
\lhead{}\chead{}\rhead{}
\lfoot{}\cfoot{\thepage}\rfoot{}

%%% SECTION TITLE APPEARANCE
\usepackage{sectsty}
%\allsectionsfont{\sffamily\mdseries\upshape} % (See the fntguide.pdf for font help)
% (This matches ConTeXt defaults)


%%% ToC (table of contents) APPEARANCE
\usepackage[nottoc,notlof,notlot]{tocbibind} % Put the bibliography in the ToC
\usepackage[titles,subfigure]{tocloft} % Alter the style of the Table of Contents
\renewcommand{\cftsecfont}{\rmfamily\mdseries\upshape}
\renewcommand{\cftsecpagefont}{\rmfamily\mdseries\upshape} % No bold!

%%% New commands%%%%%
\setcounter{secnumdepth}{4}
\titleformat{\paragraph}
{\normalfont\normalsize\bfseries}{\theparagraph}{1em}{}
\titlespacing*{\paragraph}
{0pt}{3.25ex plus 1ex minus .2ex}{1.5ex plus .2ex}

\addtocontents{loa}{\def\string\figurename{Algorithm}} % list of algorithms



%%% GRAPHICS PATH
%\graphicspath{{E:/DissertationProspectus/figures/}}

%%% END Article customizations

%%% The "real" document content comes below...

\begin{document}
	
	%Title Page
	\thispagestyle{empty}
	\begin{center}
		\begin{minipage}{\linewidth}
			\centering
			%Thesis Title
			\vspace{2cm}
			{\Huge \bf{Voice Command Browsing (Extension for Google Chrome  Browser)}\par}
			\vspace{1cm}
			%Report type		
			{\Large \bfseries{MINI PROJECT REPORT}\par}
			\vspace{0.5cm}
			{\Large \emph{Submitted in partial fulfillment of the requirements for the award of degree}\par}
			\vspace{0.5cm}
			{\Large \bf{BACHELOR OF TECHNOLOGY} \par}
			\vspace{0.5cm}
			{\Large \bf{BY} \par}
			\vspace{1cm}
			%Author's name
			{\Large \bf{RAJESH R NAIR}\hspace{3cm}     Reg.no:14015695 \par}
			{\Large \bf{UNNIKRISHNAN}\hspace{3cm}      Reg.no:14015695 \par}
			\vspace{1cm}
			% logo							
			\includegraphics[width=0.5\linewidth]{figures/logo.png}
			
			%School
			{\Large \bfseries{SCMS SCHOOL OF ENGINEERING AND TECHNOLOGY}\par}
			{\Large \emph{(Affiliated to M.G. University)}\par}
			{\Large VIDYA NAGAR, PALISSERY, KARUKUTTY\par}
			{\Large ERNAKULAM – 683 582\par}
			\vspace{1cm}
			
			%\vspace{2cm}
			
			%Date
			{\Large \monthyeardate \today}
		\end{minipage}
	\end{center}
	\clearpage
	
	
	% Certificate page
	\thispagestyle{empty}
	\begin{center}
		\begin{minipage}{\linewidth}
			\centering
			\vspace{1cm}			
			% logo		
			\includegraphics[width=0.5\linewidth]{figures/logo.png}
			\vspace{1cm}
			%School
			{\Large \bf{SCMS SCHOOL OF ENGINEERING AND TECHNOLOGY}\par}
			{\Large \emph{(Affiliated to M.G. University)}\par}
			{\Large VIDYA NAGAR, PALISSERY, KARUKUTTY\par}
			{\Large ERNAKULAM – 683 582\par}
			\vspace{1cm}
			{\Large \bf{\underline{BONAFIDE CERTIFICATE}}\par}
			\vspace{0.5cm}
			\begin{flushleft}
				This is to certify that the mini project, titled ''Voice Command Browsing (Extension for Google Chrome  Browser)'' by
			\end{flushleft} 
			%\vspace{0.5cm}
			%Author's name
			{ \bf{RAJESH R NAIR}\hspace{3cm}     Reg.no:14015695 \par}
			{ \bf{UNNIKRISHNAN}\hspace{3cm}      Reg.no:14015695 \par}
			\begin{flushleft}
				submitted in partial fulfilment of the requirement for the award of the degree of Bachelor of Technology, is a bonafide work carried under supervision, during the academic year 2015-2016.
			\end{flushleft}
			{ \bf{Ms.DEEPA SREE VARMA}\hspace{3cm}     Prof VINOD P \par}
			{ \bf{PROJECT GUIDE}\hspace{2cm}      HEAD OF DEPARTMENT \par}
			
			\vspace{2cm}
			
			{ \bf{INTERNAL EXAMINER}\hspace{3cm}     EXTERNAL EXAMINER \par}
			
			
		\end{minipage}
	\end{center}
	\clearpage
	
	
	
	
	%Abstract Page
	\thispagestyle{empty}
	\begin{center}
		\begin{minipage}{\linewidth}
			\vspace{3cm}
			{\centering \bf{Abstract}\par}
			\vspace{0.5cm}
			Put Abstract here.....
		\end{minipage}
	\end{center}
	\clearpage
	
	
	%ACKOWLEDGEMENT Page
	\thispagestyle{empty}
	\begin{center}
		\begin{minipage}{\linewidth}
			\vspace{3cm}
			{\centering \bf{ACKOWLEDGEMENT}\par}
			\vspace{0.5cm}
			We are greatly indebted to Prof.M.Madhavan, Principal, SSET, Ernakulam and Prof.Vinod, Head of department, Department of Computer Science and Engineering, SSET, who whole heartedly granted us the permission to carry out the mini project.
			We would like to thank our guide, Ms.Deepa Sree Varma, Assistant Professor, Department of  Computer Science and Engineering, SSET who has given us valuable guidance and support throughout the project. Also, we would like to thank our project coordinators, Ms.Shilpa  P C and Ms.Gayathri  Assistant Professors, Department of Computer Science and Engineering, SSET, who supported and instructed us all the way.
			We would like to express our sincere gratitude to all the teachers of Computer Science Department who gave us moral and technical support through the course of our mini project. We would like to thank the supporting staff in the Computer lab whose dedicated work kept the lab working smoothly, thus ensuring our time at the lab went hassle free.
			
		\end{minipage}
	\end{center}
	\clearpage
	
	\tableofcontents{}
	\newpage

\chapter{Introduction}
\label{chap:Intro}



\section{OVERVIEW}
\label{sec:OVERVIEW}



\section{PROBLEM ANALYSIS}
\label{sec:PROBLEM ANALYSIS}

Problem analysis is the process of understanding the actual problems, user needs and proposing solutions to meet those needs. The goal of problem analysis is to gain a better understanding of the problem being solved before development begins. It is the process of gathering and interpreting facts, diagnosing problems and using the information to recommend improvements on the system. Problem analysis is problem solving activities that require intensive communication between users and the system developers. A problem can be defined as the difference between things as perceived and things as desired. The system is studied and analyzed. The system is viewed as a whole and the input to the system are identified. The output from the system is given to various processes.


\section{EXISTING SYSTEM}
\label{sec:EXISTING SYSTEM}



\section{PROPOSED SYSTEM}
\label{sec:PROPOSED SYSTEM}


\section{FEASIBILITY STUDY}
\label{sec:FEASIBILITY STUDY}

Feasibility study is a procedure that identifies, describes and evaluates candidate systems and selects the best system for the job. An estimate is made whether the identified users need may be satisfied using the current software and hardware technologies. The study will decide whether the proposed system will be cost effective from a business point of view and if it can be developed using the given existing budgetary constraints. 
The key considerations involved in the feasibility analysis are the following:


\begin{enumerate}
	\item Economic feasibility
	\item Technical feasibility
	\item Operational feasibility
\end{enumerate}


\subsection{ECONOMIC FEASIBILITY}
\label{subsec:ECONOMIC FEASIBILITY}

Economic study is the most frequently used method for evaluating the effectiveness of candidate system. More commonly known as cost/benefit analysis, the procedure is to determine the benefits and savings that are accepted from a candidate system and compares with costs. If benefit outweighs cost, then decisions are made to design and implement the system. Otherwise further alterations will have to be made if to have a chance of being approved.
Less hardware is required and can also be mounted on the existing wheelchair with reduced complexity.Hence this project is economically feasible and is cost effective because of its compatibility and effort saving nature.


\subsection{TECHNICAL FEASIBILITY}
\label{subsec:TECHNICAL FEASIBILITY}

Technical feasibility is a measure of how feasible the project is technically. The effort and technology included in the conventional system is not needed as the whole process is automated.
The hierarchy of the new system is very easier than the existing system. The new system is very much easier and user friendly. Operational cost is very easy. The maintenance and modification of the new system needs very less human effort. 

\chapter{DESIGN}
\label{cha:DESIGN}

\section{BLOCK DIAGRAM}
\label{sec:BLOCK DIAGRAM}

\section{BLOCK DIAGRAM}
\label{sec:DATA FLOW DIAGRAMS}

\subsection{CONTEXT LEVEL DFD}
\label{subsec:CONTEXT LEVEL DFD}

\begin{figure}[h]
	\centering
	\includegraphics[width=0.5\linewidth]{figures/context_level_DFD.png}
	\caption{Context level DFD}
	\label{fig:Context level DFD}
\end{figure}


\begin{enumerate}
	\item Context level DFD is the most basic representation of the system.
	\item This indicates the basic working of the system.
	\item The user controls the application.
\end{enumerate}


An extensive literary survey of subject will be collected from 

\begin{itemize}
	\item Relevant Ayurveda texts.
	\item Modern medicine texts.
	\item Contemporary journals.
	\item Electronic search (using Pub Med, Google Scholar and Web of Science).
	\item Other related sources.
\end{itemize}


%%%%%%%%%%%%%%%% TO BE COMPLETED%%%%%%%%%%%%%%%%%%%%%%%%%%%

\textbf{Study Design :} Invitro cell Line Study.

\textbf{Study period :}  18 months.

\textbf{Study population :} Doxorubicin treated H9c2 cardiomyoblasts

\textbf{Sample size :} Not applicable

\textbf{Study Setting : } 

\begin{itemize}
	\item[1] \textbf{Drug preparation:}\\
	Department of Rasa sastra \&  Bhaishajya kalpana   \\           	             Govt.Ayurveda College, Tripunithura
	\item[2] \textbf{Vitro cell line study} in Biogenix Research \\
	Centre,Trivandrum
	\item[3] \textbf{Analytical study:}\\
	Department of Rasa sastra \& Bhaishajya kalpana \\
	Govt.Ayurveda College, Tripunithura
\end{itemize}

\textbf{Inclusion Criteria :} Not applicable 

\textbf{Exclusion Criteria :} Not applicable 
	
\textbf{Sampling Technique :} Not applicable 

\textbf{Data Collection    :} Primary data collected from lab experiments\\


\textbf{Analytical Study:}

\begin{enumerate}
	\item Organoleptic Characters
	\begin{itemize}
		\item colour
		\item odour
		\item taste
	\end{itemize}
	\item Physicochemical Analysis
	\begin{itemize}
		\item Kashaya and Ksheerapaka
		\begin{itemize}
			\item[-] pH
			\item[-] HPTLC
			\item[-] Specific gravity
			\item[-] Total ash
			\item[-] Acid insoluble ash
		\end{itemize}
		\item Arka
		\begin{itemize}
			\item[-] pH
			\item[-] Specific gravity
			\item[-] HPTLC
		\end{itemize}
	\end{itemize}
\end{enumerate}

\textbf{Study Tools:}

\begin{enumerate}
	\item Cytotoxicity assay by MTT method
	\item Cytotoxicity assay by Direct microscopic observation.
	\item Mitopotential flow cytometry
	\item Annexin V/FITC flow cytometry
	\item DCFDA Staining
\end{enumerate}

\subsection{Procedure}
\label{subsec:Procedure}

\subsubsection{Cell culture and Treatment}
\label{subsubsec:Cell culture and Treatment}

H9C2 (cardiomyoblast cell line) was initially procured from National Centre for Cell Sciences (NCCS), Pune, India and maintained in Dulbecos modified Eagles medium ( Gibco, Invitrogen).

The  cell line was  cultured in 25 cm$^2 $ tissue culture flask with DMEM supplemented with 10\% FBS, L-glutamine, sodium bicarbonate and antibiotic solution containing: Penicillin (100U$\mu$g/ml), Streptomycin (100$\mu$g/ml), and Amphotericin B (2.5$\mu$g/ml). Cultured cell lines were kept at 37$^{\circ}$C in a humidified 5\% CO$_2$ incubator (NBS Eppendorf, Germany). 

The viability of cells were evaluated by direct observation of cells by Inverted phase contrast microscope and followed by MTT assay method.

\subsubsection{Cells seeding in 96 well plate}
\label{subsubsec:Cells seeding in 96 well plate}

Two days old confluent monolayer of cells were trypsinized and the cells were suspended in 10\% growth medium, 100$\mu$l cell suspension ($ 5\times104 $ cells/well) was seeded in 96 well tissue culture plate and incubated at 37$^{\circ}$C in a humidified 5\% CO$_2$ incubator. 

\subsubsection{Dosage Optimization Study}
\label{subsub:Dosage Optimization Study}

Nontoxic concentrations of various kalpanas of dashamoola will be determined by standard
MTT Assay in H9C2 cardiomyoblast cells. Briefly, different concentrations of samples such as
1.75$\mu$l, 3.75$\mu$l, 6.5$\mu$l, 12.5$\mu$l, 25$\mu$l, 50$\mu$l, 100$\mu$l volumes of extracts were added to 70\% confluent H9C2 cells, incubated for 24 hours and viability was determined by MTT Assay. Morphological changes were recorded using a phase contrast microscopy (Olympus CKX41 20X magnification). LD50 values will be calculated using ED50 plus version 1.0 and sublethal concentrations were used for further studies.

\subsubsection{ Preparation of compound stock}
\label{subsubsec: Preparation of compound stock}

After that the extract solution was filtered through 0.22 $\mu$m Millipore syringe filter to
ensure the sterility.

Doxorubicin was used to induce toxicity as per methods described by Xiao et al, 2012. After 24
hours the growth medium was removed, Doxorubicin (Sigma Aldrich, US) was added at a final
concentration of 0.1\% to induce toxicity and incubated for an hour.

Sublethal concentrations of the kalpanas previously filtered through 0.22$\mu$m Millipore syringe filter was added to the 70\% confluent cells.

The nontoxic concentrations are to be added in triplicates to respective wells of cells and
incubated at 37$^{\circ}$C in a humidified 5\% CO$ ^{2} $ incubator.

\subsubsection{Study tools}
\label{subsubsec:Study tools}

\begin{itemize}
	\item \textbf{Cytotoxicity Assay by Direct Microscopic observation:} \\
	Entire plate was observed at after 24 hours in an inverted phase contrast tissue culture microscope (Olympus CKX41 with Optika Pro5 CCD camera) and microscopic observation were recorded as images.  Any detectable changes in the morphology of the cells, such as rounding or shrinking of cells, granulation and vacuolization in the cytoplasm of the cells were considered as indicators of cytotoxicity.  
	\item \textbf{Cytotoxicity Assay by MTT Method:} \\
	The purpose of this assay is to essentially measure the number of metabolically active cells in a 96-well plate.
	
	Fifteen mg of MTT (Sigma, M-5655) was reconstituted in 3 ml PBS until completely dissolved and sterilized by filter sterilization. After 24  hours of incubation period, the sample content in wells were removed and 30$\mu$l of reconstituted MTT solution was added to all test and cell control wells, the plate was gently shaken well, then incubated at 37$^{\circ}$C in a humidified 5\% CO$_2 $ incubator for 4 hours. After the incubation period, the supernatant was removed and 100$\mu$l of MTT Solubilization Solution ( DMSO was added and the wells were mixed gently by pipetting up and down  in order to solubilize the formazan crystals. The absorbance values were measured by using microplate reader at a wavelength of 570 nm (\textit{Laura B. Talarico et al., 2004}).
	
	The percentage of growth inhibition was calculated using the formula:
	
	\begin{equation}
	\label{eqn:viability}
	\%\ of\ viability = \frac{Mean\ OD\ Samples \times 100}{Mean\ OD\ of\ control\ group}
	\end{equation}
	
	\item \textbf{Mitopotential Flow cytometry:} to check the opening of the mitochondrial permeability transition pore (PTP).
	\item \textbf{APOPTOSIS – Measurement by ANNEXIN V/FITC FLOW CYTOMETERY} \\
	To assess the rate of apoptosis
	\item \textbf{ROS GENERATION – By DCFDA STAINING} \\
	To measure the intracellular ROS(Reactive Oxygen Species) production.
	\item \textbf{Method of Preparation – ANNEXURE 1}
\end{itemize}

\section{Outcome Variable}
\label{sec:Outcome Variable}

Cardioprotective activity in doxorubicin induced cardiomyopathy on H9c2 cardiomyoblasts by MTT method, direct microscopic observation, mitopotential flow cytometry, Annexin V/FITC flow cytometry, DCFDA staining.


\section{Statistical Analysis}
\label{sec:Statistical Analysis}

Appropriate statistical technique, if necessary, for the description and summarization of data will be adopted.

\section{Results and conclusion}
\label{sec:Results and conclusion}

The outcome of the work will be concluded after discussion.

\section{Ethical Consideration}
\label{Sec:ethical}

Not Applicable




%%%%%%%%%%%%%%%%%% GLOSSARIES %%%%%%%%%%%%%%%%%%%%%%%%%%%%%%%%%%%

\section{Glossary}
\label{sec: Glossary}

\begin{itemize}
	\item \textbf{Cardio Protective Agent - } Any protective agent that is able to prevent damage to the heart.
	\item \textbf{H9c2 cardiomyoblast - } A sub clone of the original clonal cell line derived from embryonic rat heart tissue.
	\item \textbf{Cell line - }A cell culture developed from a single cell and having uniform genetic composition. \\
	Cell lines provide a pure population of cells which is valuable since it produces a    consistent sample and reproducible results.
	\item \textbf{Antioxidants - } A substance (such as beta-carotene or vitamin C) that inhibits oxidation or reactions promoted by oxygen, peroxides, or free radicals.	
	\item \textbf{MTT Assay - } It is a colorimetric assay for assessing cell metabolic activity.
	\item \textbf{Mitopotential Flow Cytometry - } An assay to detect the loss of mitochondrial membrane potential that occurs during apoptosis.
	\item \textbf{ROS - } Reactive Oxygen Species is a phrase used to describe a number of reactive molecules and free radicals derived from molecular oxygen.
	\item \textbf{Apoptosis - } Process of programmed cell death that occurs in multicellular organisms.
\end{itemize}



\section*{ANNEXURE 1}

Physico chemical analysis is done and from this the required samples are taken.


%%%%%%%%%%%%%%%%%%%%%%% TABLE%%%%%%%%%%%%%%%%%%%%%%%%%%%%%%%%%%%%%%%%%%%%%%%%%%%

\begin{table}[h]
	\centering
	\caption{Drugs and their Quantity}
	\label{tab:Raw Drug}
	\begin{tabular}{|c|c|c|}
		\hline
		RAW DRUG                            & SCIENTIFIC NAME          & QUANTITY         \\ \hline
		Bilwa                               & Aegle marmelos           & 1 part   \\ \hline
		Agnimantha                          & Premna mucronata         & 1 part   \\ \hline
		Shyonaka                            & Oroxylum indicum         & 1 part   \\ \hline
		Gambhari                            & Gmelina arborea          & 1 part   \\ \hline
		Paatala                             & Stereospermum suaveolens & 1 part   \\ \hline
		Shaalaparni                         & Desmodium gangeticum     & 1 part   \\ \hline
		Prishniparni                        & Uraria picta             & 1 part   \\ \hline
		Gokshura                            & Tribulus terrestris      & 1 part   \\ \hline
		Brihathi                            & Solanum indicum          & 1 part   \\ \hline
		Kantakaari                          & Solanum xanthocarpum     & 1 part   \\ \hline
	\end{tabular}
\end{table}

%%%%%%%%%%%%%%%%%%%%%%%%%%%%%%%%%%%%%%%%%%%%%%%%%%%%%%%%%%%%%%%%%%%%%%%%%%%%%%%%%%%%%%

\subsection*{METHOD OF PREPARATION:}

\begin{enumerate}
	\item \textbf{Dashamoola sritha Kashaya}\\
	Roots of the medicinal plant group is crushed and 16 times water is to be added. It is then heated and reduced to 1/8 th . The decoction thus obtained is filtered and used.
	\item \textbf{Dashamoola Ksheerapaka}\\
	Roots of the medicinal plant group is crushed and put into a mixture of 8 times milk and 32 times water. It is then boiled and reduced to the amount of milk.
	\item \textbf{Dashamoola Arka}\\
	The drugs are cleaned and coarsely powdered. Ten times water should be added to it and is
	soaked for 24 hrs. Then Arka is extracted by using Arka yantra.
\end{enumerate}

%\begin{itemize}
%	\item The ingredients are added with 5 adhaka(15 l) of water. It is boiled till yava is cooked properly.
%	\item Then the decoction is filtered and added with cooked harithaki, 8 palas(384 g) of ghrita and thaila, and 1 tula (4.8kgs) of guda.
%\end{itemize}
%
%Chitraka Shodhana: Chitraka is used after shodhana for the preparation of Agasthya hareethaki.  
%
%\begin{itemize}
%	\item Roots of chitraka are cut into smaller pieces and are kept immersed in choornodaka for 24 hours. Later these roots are taken out, washed with warm water and dried.
%	\item Preparation of Choornodaka(Rasa tarangini):
%	\item 2 ratti(250mg) of churna is added with 5 tola(60 ml) of water and left undisturbed for 3 yama (9 hrs). Later it is filtered and stored in a glass bottle.
%\end{itemize}
%
%Avaleha is prepared according to Avaleha kalpana in Sharangadhara Samhita. After cooling, add 1 kudava (192g) each of honey and pippali.
\begin{appendix}    
	%\listoffigures
	\listoftables
\end{appendix}

%\nocite{Charaka}
%\nocite{Ayurvedic}
%\nocite{Susrutha}
%\nocite{Chikitsa}
%\nocite{Shodala}
%\nocite{Arka}
%\bibliographystyle{unsrt}
%\bibliography{synopsis}

\end{document}